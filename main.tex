\documentclass [xcolor=svgnames, t] {beamer}
\usepackage[utf8]{inputenc}
\usepackage{lmodern}
\usepackage{booktabs, comment}
\usepackage[absolute, overlay]{textpos}
\usepackage{pgfpages}
\usepackage[font=footnotesize]{caption}
\usepackage{tikz}
\usepackage[autostyle]{csquotes}
\usepackage{amsmath}
\usepackage[makeroom]{cancel}
\usepackage{textpos}
\usepackage{pgfplots}
\usepackage[vietnamese]{babel}
\usepackage[ruled]{algorithm2e}
\usepackage{caption}
\usepackage{multicol}

\useoutertheme{infolines}

\definecolor{gold}{RGB}{254, 206, 0}

\setbeamercolor{title in head/foot}{bg=gold, fg=black}
\setbeamercolor{author in head/foot}{bg=myuniversity}
\setbeamertemplate{page number in head/foot}{}

\pgfplotsset{compat=1.18}

\usetikzlibrary{decorations}
\usetikzlibrary{decorations.pathreplacing}
\usetikzlibrary{decorations.pathreplacing,calligraphy}
\usetikzlibrary{arrows.meta}
\usetikzlibrary{quotes}
\usetikzlibrary{intersections}
\usetikzlibrary{calc}
\usetikzlibrary{arrows}
\pagestyle{empty}

\SetFuncSty{textsc}
\SetDataSty{emph}
\SetCommentSty{commentsty}
\SetKwComment{Comment}{$\triangleright$ }{}
\SetKwBlock{Block}{function}{}
\SetKwInOut{Input}{input}
\SetKwInOut{Output}{output}
\SetKw{Return}{return}
\SetKwData{Node}{node}
\SetKwData{Leaf}{leaf}
\SetKwFunction{Build}{build}
\SetKwFunction{Variance}{variance}
\SetKwFunction{Median}{median}
\SetKwFunction{Split}{split}


\usetheme{Madrid}
\definecolor{myuniversity}{RGB}{0, 85, 150}
\usecolortheme[named=myuniversity]{structure}


\theoremstyle{definition}
% \setcounter{MaxMatrixCols}{30}
% \newtheorem{theorem}{Theorem}[section]
\newtheorem{acknowledgement}[theorem]{Acknowledgement}
% \newtheorem{algorithm}[theorem]{Algorithm}
\newtheorem{axiom}[theorem]{Axiom}
\newtheorem{case}[theorem]{Case}
\newtheorem{claim}[theorem]{Claim}
\newtheorem{conclusion}[theorem]{Conclusion}
\newtheorem{condition}[theorem]{Condition}
\newtheorem{conjecture}[theorem]{Conjecture}
% \newtheorem{corollary}[theorem]{Corollary}
\newtheorem{criterion}[theorem]{Criterion}
% \newtheorem{definition}{Definition}
% \newtheorem{example}{Example}
\newtheorem{exercise}[theorem]{Exercise}
% \newtheorem{lemma}[theorem]{Lemma}
\newtheorem{notation}[theorem]{Notation}
% \newtheorem{problem}[theorem]{Problem}
\newtheorem{proposition}[theorem]{Proposition}
\newtheorem{remark}{Remark}
% \newtheorem{solution}[theorem]{Solution}
\newtheorem{summary}[theorem]{Summary}

\title[RUNNING TITLE]{Đường cong Elliptic}
\subtitle{(trước và sau quantum)}
\institute[]{School of Computer Science  \\University of Windsor}
\titlegraphic{\includegraphics[height=2.5cm]{UWin.jpeg}}
\author[Main Author]{
	Main Author ,
	2nd Author,
	3rd Author }


\institute[]{School  of Computer Science  \\University of Windsor}
\date{\today}


\addtobeamertemplate{navigation symbols}{}{%
    \usebeamerfont{footline}%
    \usebeamercolor[fg]{footline}%
    \hspace{1em}%
    \insertframenumber/\inserttotalframenumber
}

\begin{document}
\begin{frame}
    \maketitle
\end{frame}


\logo{\includegraphics[scale=0.15]{UWin.jpeg}~%
}





\begin{frame}
    \frametitle{Table of Contents}
    \tableofcontents
\end{frame}

\section{Introduction}
\begin{frame}{Introduction}
    \begin{definition}
        \label{define:1.1}
        Ta định nghĩa bài toán logarit rời rạc trên một nhóm với phép nhân các số nguyên theo modulo $p$, $\mathbb{Z}_{/\mathbb{Z}_p}$, như sau:

        Cho $g, a \in \mathbb{Z}_{/\mathbb{Z}_p}$, với $a$ là phần tử của nhóm cyclic có phần tử sinh $g$, tìm số nguyên $k$ thỏa mãn:
        \begin{equation}
            \label{equation:1.1}
            g^k \equiv a \pmod{p}
        \end{equation}
    \end{definition}
    \begin{definition}
        \label<asdasdasd>{define:1.2}
        Ta định nghĩa bài toán phân tích rời rạc như sau: Cho một số nguyên $N$, có ước là hai số nguyên tố lớn $p$ và $q$. Tìm $p$ và $q$.
    \end{definition}
\end{frame}
\section{Preliminaries}
\begin{frame}{Preliminaries}
    \cite{Vishwanath10}
    \begin{definition}
        \label{define:2.1}
        Một \textit{nhóm} là một cấu trúc đại số bao gồm:
        \begin{itemize}
            \item Một tập phần tử $G$
            \item Một toán tử đóng $(\cdot)$ trên tập $G$ thỏa mãn tính chất kết hợp. Nghĩa là $(a \cdot b) \cdot c  = a \cdot (b \cdot c)$, với mọi $a, b, c \in G$.
            \item Một phần tử đơn vị $1$, để $a \cdot 1 = a$ với mọi $a \in G$.
            \item Tồn tại phần tử nghịch đảo $a^{-1} \in G$ nếu $a \in G$ để $a^{-1} \cdot a = 1$.
        \end{itemize}

        Một \textit{nhóm} thỏa mãn thêm điều kiện toán tử hai ngôi có thêm tính giao hoán (hay $a \cdot b = b \cdot a$) là nhóm giao hoán hoặc \textit{nhóm Abel}
    \end{definition}
\end{frame}
\begin{frame}{Preliminaries}
    \begin{example}
        Với tập số nguyên $\mathbb{Z}$ được trang bị toán tử $+$ và xem số $0$ như phần tử đơn vị,
        xem các cặp số nguyên có dấu ngược nhau là các cặp nghịch đảo, ta có một \textit{nhóm}.

        Ngược lại, tập số tự nhiên $\mathbb{N}$ không phải một nhóm vì không định nghĩa được phần tử nghịch đảo.
    \end{example}
\end{frame}

\begin{frame}{Preliminaries}
    \begin{definition}
        \label{define:2.2}
        Cho nhóm $G$ được trang bị toán tử $(\cdot)$ với phần tử đơn vị $1$. Với mỗi $a \in G$, \textit{bậc} của $a$, là số nguyên $n$ nhỏ nhất thỏa mãn:
        \begin{equation}
            \label{equation:2.1}
            \underbrace{a \cdot a \cdot a \cdot \ldots \cdot a}_{n} = 1
        \end{equation}
        Tập $\{a, a^2, a^3, a^4, \ldots, a^n\}$ là nhóm cyclic con của $G$ có bậc $n$, $a$ được gọi là \textit{phần tử sinh} của nhóm đó.
    \end{definition}
\end{frame}
\begin{frame}{Preliminaries}
    \begin{definition}
        \label{define:2.3}
        Một \textit{trường} là một cấu trúc đại số bao gồm:
        \begin{itemize}
            \item Tập $G$ là tập đóng dưới phép cộng và phép nhân.
            \item Phép cộng và phép nhân phải có tính kết hợp trên tập $G$.
            \item Phép cộng và phép nhân phải có tính giao hoán trên tập $G$.
            \item Tồn tại phần tử nghịch đảo cho cả phép cộng và phép nhân.
            \item Phép nhân có tính chất phân phối đối với phép cộng: $a \cdot (b + c) = (a \cdot b) + (a \cdot c)$.
            \item Phần tử đơn vị của phép cộng và phép nhân phải khác nhau.
        \end{itemize}
    \end{definition}
\end{frame}
\begin{frame}{Preliminaries}
    \begin{definition}
        \textit{Tính đặc trưng} của trường $F$ là số $n$ nhỏ nhất thỏa mãn tổng của $n$ phần tử $1$ bằng $0$. Kí hiệu $char(F) = n$.
    \end{definition}
    \begin{example}
        Nếu tính đặc trưng của $F$, $char(F)$, là 2 và phần tử đơn vị là $1$, thì $ 1 + 1 = 0 $.

        Nếu $char(F) = 3$ thì $ 1 + 1 + 1 = 0 $.
    \end{example}

\end{frame}
\begin{frame}{Preliminaries}
    \begin{definition}
        \textit{Trường Galois} là trường bao gồm một tập hữu hạn phần tử.
    \end{definition}
    \begin{example}
        Tập số nguyên theo modulo số nguyên tố $p$, $\mathbb{Z}_{/\mathbb{Z_p}}$ là một trường Galois. Với $p$ phần tử, $0$ đến $p-1$,
        kí hiệu $GF(p)$.
    \end{example}
    Phần còn lại, ta sẽ đề cập nhiều hơn đến trường của số nguyên tố $GF(p)$, tổng quát hơn là $GF(p^n)$.
\end{frame}
\section{Elliptic Curve Cryptography}
\subsection{Basic}
\begin{frame}{Basic}
    \begin{definition}
        % \label{define:3.1}
        Một đường cong elliptic $E(F)$ là một tập điểm trên trường $F$, thỏa mãn phương trình có dạng:
        \begin{equation}
            \label{equation:3.1}
            y^2 + a_1xy + a_2y = x^3 + a_3x^2 + a_4x + a_5
        \end{equation}
        Trong đó $a_1, a_2, a_3, a_4, a_5 \in F$.
    \end{definition}
    Nếu ta giả sử tính đặc trưng của $F$ khác 2, hay $ 1 + 1 = 2 \neq 0 $, phương trình có thể viết lại dưới dạng:
    \begin{equation}
        \label{equation:3.2}
        y^2 = x^3 + a_3x^2 + a_4x + a_5
    \end{equation}
    Phương trình có thể rút gọn hơn nữa nếu tính đặc trưng của trường khác 3, ta được phương trình đơn giản hơn, được gọi là phương trình Weierstrass.
    \begin{equation}
        \label{equation:3.3}
        y^2 = x^3 + ax + b \mid_{a=a_4, b = a_5}
    \end{equation}

\end{frame}
\begin{frame}
    \frametitle{Đồ thị}
    \begin{figure}
        \label{figure:1}
        \caption{Đường cong Elliptic $E$ trên $\mathbb{R}^2$}
        \begin{tikzpicture}[scale = 0.6, line cap=round,line join=round,>=triangle 45,x=1cm,y=1cm]
            \begin{axis}[
                    x=1cm,y=1cm,
                    axis lines=middle,
                    xmin=-5.912600967627929,
                    xmax=5.112504977805202,
                    ymin=-5.000000419005899,
                    ymax=5.000000419005899,
                    legend pos=north west,
                    xticklabels={,,},
                    yticklabels={,,},
                    % xtick={-5,-4,...,5},
                    % ytick={-5,-4,...,5},
                ]
                \clip(-5.912600967627929,-5.000000419005899) rectangle (10.112504977805202,5.000000419005899);
                \draw[line width=0.5pt,color=blue,smooth,samples=100,domain=-2.1038033040947663:10.112504977805202] plot(\x,{sqrt(abs((\x)^(3)-3*(\x)+3))});
                \draw[line width=0.5pt,color=blue,smooth,samples=100,domain=-2.1038033040947663:10.112504977805202] plot(\x,{0-sqrt(abs((\x)^(3)-3*(\x)+3))});
                % \legend{$E: y^2 = x^3 -3x + 3$}
                \addlegendimage{line width=0.3mm, color=blue}
                \addlegendentry{$E: y^2 = x^3 -3x + 3$}
            \end{axis}
        \end{tikzpicture}
    \end{figure}
\end{frame}

\begin{frame}
    \frametitle{Đồ thị}
    \begin{figure}
        \label{figure:2}
        \caption{Cộng hai điểm $P$ và $Q$ trên $E$}
        \begin{tikzpicture}[scale = 0.6, line cap=round,line join=round,>=triangle 45,x=1cm,y=1cm]
            \begin{axis}[
                    x=1cm,y=1cm,
                    axis lines=middle,
                    xmin=-5.912600967627929,
                    xmax=5.112504977805202,
                    ymin=-5.000000419005899,
                    ymax=5.000000419005899,
                    legend pos=north west,
                    xticklabels={,,},
                    yticklabels={,,},
                    % xtick={-5,-4,...,5},
                    % ytick={-5,-4,...,5},
                ]
                \clip(-5.736430725533388,-5.733053262983473) rectangle (5.923722428745249,5.872549467181578);
                \draw[line width=0.5pt,color=blue,smooth,samples=100,domain=-2.103803291543122:5.923722428745249] plot(\x,{sqrt(abs((\x)^(3)-3*(\x)+3))});
                \draw[line width=0.5pt,color=blue,smooth,samples=100,domain=-2.103803291543122:5.923722428745249] plot(\x,{0-sqrt(abs((\x)^(3)-3*(\x)+3))});
                \draw [line width=0.5pt,domain=-5.736430725533388:5.923722428745249] plot(\x,{(--2.3486301286669415-0.40680839258599555*\x)/1.3559822370127241});
                \draw [line width=0.2pt,dotted] (1.4459883083356893,-1.9982396827947813)-- (1.4459883083356893,1.9982396827947813);
                \begin{scriptsize}
                    \draw[color=black] (-1.9451762496392822,0.28794979855338965) node {$E$};
                    \draw [fill=black] (0,1.7320508075688772) circle (1.5pt);
                    \draw[color=black] (0.2232031088756924,1.9926505521028977) node {$Q$};
                    \draw [fill=black] (-1.3559822370127241,2.1388592001548727) circle (1.5pt);
                    \draw[color=black] (-1.1405574939639143,2.4017787329547793) node {$P$};
                    \draw[color=black] (-5.463678604965468,2.8700487609709) node {$L$};
                    \draw [fill=black] (1.4459883083356893,1.2982396827947813) circle (1.5pt);
                    \draw[color=black] (1.777890196112844,1.5698847652226198) node {$\ -R$};
                    \draw [fill=black] (1.4459883083356893,-1.2982396827947813) circle (1.5pt);
                    \draw[color=black] (2.4324952854758553,-1.0348979862010284) node {$R$};
                \end{scriptsize}
                \addlegendimage{line width=0.3mm, color=blue}
                \addlegendentry{$E: y^2 = x^3 -3x + 3$}
            \end{axis}
        \end{tikzpicture}
    \end{figure}
\end{frame}

\begin{frame}
    \frametitle{Đồ thị}
    \begin{figure}
        % \label{figure:2}
        \caption{Cộng hai điểm $P$ và $P$ trên $E$}
        \begin{tikzpicture}[scale = 0.55, line cap=round,line join=round,>=triangle 45,x=1cm,y=1cm]
            \begin{axis}[
                    x=1cm,y=1cm,
                    axis lines=middle,
                    xmin=-6.912600967627929,
                    xmax=6.112504977805202,
                    ymin=-6.000000419005899,
                    ymax=5.900000419005899,
                    legend pos=north west,
                    xticklabels={,,},
                    yticklabels={,,},
                    % xtick={-5,-4,...,5},
                    % ytick={-5,-4,...,5},
                ]
                \clip(-5.736430725533388,-5.733053262983473) rectangle (5.923722428745249,5.872549467181578);
                \draw[line width=0.8pt,color=blue,smooth,samples=100,domain=-2.103803291543122:5.923722428745249] plot(\x,{sqrt(abs((\x)^(3)-3*(\x)+3))});
                \draw[line width=0.8pt,color=blue,smooth,samples=100,domain=-2.103803291543122:5.923722428745249] plot(\x,{0-sqrt(abs((\x)^(3)-3*(\x)+3))});
                \draw [line width=0.8pt,domain=-5.736430725533388:5.923722428745249] plot(\x,{(-0.011820467984078764-0.0023798531917433863*\x)/-0.0040177629872759635});
                \draw [line width=0.8pt,dotted] (3.06684049847989,-5.458642606222206)-- (3.06684049847989,5.458642606222206);
                \begin{scriptsize}
                    % \draw[color=black] (-1.9451762496392822,0.28794979855338965) node {$E$};
                    \draw [fill=black] (-1.36,2.1364793469631294) circle (2pt);
                    \draw [fill=black] (-1.3559822370127241,2.1388592001548727) circle (2pt);
                    \draw[color=black] (-1.2405574939639143,2.4017787329547793) node {$P$};
                    % \draw[color=black] (-5.463678604965468,-0.10754077627009614) node {$L$};
                    \draw [fill=black] (3.06684049847989,4.758642606222206) circle (2pt);
                    \draw[color=black] (3.400765313491976,4.520199090406824) node {$\ -R$};
                    \draw [fill=black] (3.06684049847989,-4.758642606222206) circle (2pt);
                    \draw[color=black] (4.0553704028549875,-4.4988499174136285) node {$R$};
                \end{scriptsize}
                \addlegendimage{line width=0.3mm, color=blue}
                \addlegendentry{$E: y^2 = x^3 -3x + 3$}
            \end{axis}
        \end{tikzpicture}
    \end{figure}
\end{frame}

\begin{frame}
    \frametitle{Đồ thị}
    \begin{figure}
        % \label{figure:2}
        \caption{Cộng hai điểm $P$ và $-P$ trên $E$}
        \begin{tikzpicture}[scale = 0.55, line cap=round,line join=round,>=triangle 45,x=1cm,y=1cm]
            \begin{axis}[
                    x=1cm,y=1cm,
                    axis lines=middle,
                    xmin=-6.912600967627929,
                    xmax=6.112504977805202,
                    ymin=-6.000000419005899,
                    ymax=5.900000419005899,
                    legend pos=north west,
                    xticklabels={,,},
                    yticklabels={,,},
                    % xtick={-5,-4,...,5},
                    % ytick={-5,-4,...,5},
                ]
                \clip(-5.083985307432931,-5.419798893648197) rectangle (4.940270977414931,5.225844038096666);
                \draw[line width=0.8pt,color=blue,smooth,samples=100,domain=-2.103803385261131:4.940270977414931] plot(\x,{sqrt(abs((\x)^(3)-3*(\x)+3))});
                \draw[line width=0.8pt,color=blue,smooth,samples=100,domain=-2.103803385261131:4.940270977414931] plot(\x,{0-sqrt(abs((\x)^(3)-3*(\x)+3))});
                \draw [->,line width=0.8pt] (-1.3559822370127241,-5.8388592001548725) -- (-1.3559822370127241,5.138859200154872);
                \begin{scriptsize}
                    % \draw[color=black] (-1.9653277965913738,0.24888872474237672) node {$E$};
                    \draw [fill=black] (-1.3559822370127241,2.1388592001548727) circle (2pt);
                    \draw[color=black] (-1.0929001609366532,2.409827452774213) node {$P$};
                    \draw [fill=black] (-1.3559822370127241,-2.1388592001548727) circle (2pt);
                    \draw[color=black] (-1.0929001609366532,-2.409827452774213) node {$-P$};
                    \draw[color=black] (-0.7460030555104643,5.020669201857089) node {$\mathcal{O}$};
                    % \draw[color=black] (-1.1680770043461637,1.292349320475078) node {$\ L$};
                \end{scriptsize}
                \addlegendimage{line width=0.3mm, color=blue}
                \addlegendentry{$E: y^2 = x^3 -3x + 3$}
            \end{axis}
        \end{tikzpicture}
    \end{figure}
\end{frame}

\begin{frame}
    \frametitle{Tính $P+Q$ nếu $P(x_1,y_1)$ và $Q(x_2,y_2)$ phân biệt  }
    Để sau


\end{frame}
\begin{frame}
    \frametitle{Tính $P+Q$ nếu $P(x_1,y_1)$ và $Q(x_2,y_2)$ phân biệt  }
    Sử dụng phép cộng này, ta định nghĩa phép \textit{nhân vô hướng} trên nhóm $G$
    \begin{definition}
        Phép \textit{nhân vô hướng} trên nhóm $G$ là phép cộng một điểm nhiều lần.
        \begin{equation}
            nP = \underbrace{P+P+ \ldots +P}_{n}
        \end{equation}
    \end{definition}
\end{frame}
\begin{frame}
    \frametitle{Thuật toán nhân đôi và cộng}

    \begin{algorithm}[H]
        \Input {$n, P$}
        \Output{$R = nP$}
        \Begin
        {
            $Q \gets P$ \\
            $E \gets \mathcal{O}$ \\
            \While{$n>0$}
            {
                \eIf{$n \equiv 1 \pmod{2}$}
                {
                    $R \gets R + Q$
                }
                {
                    $R \gets R + Q$ \\
                    $Q \gets 2 \cdot Q$ \\
                    $n \gets \lfloor \frac{n}{2} \rfloor $
                }
            }

            \Return $R$
        }
        \caption{Nhân đôi và cộng}
    \end{algorithm}

\end{frame}

\begin{frame}{Frame Title}
    \begin{corollary}
        Vì $nP + mP = \underbrace{P+P+P+\ldots+P}_{n} + \underbrace{P+P+P+\ldots+P}_{m} = (n+m)P$
        nên khi cộng 2 bội của $P$ ta thu được bội khác của $P$. Tập hợp các bội của $P$ là một nhóm cyclic với bậc $k$, trong đó $P$ là phần tử sinh.
    \end{corollary}
    \begin{theorem}[Định lý Lagrange]
        Nếu $H$ là nhóm con của nhóm hữu hạn $G$, thì bậc (số phần tử) của $G$ chia hết cho bậc của $H$.
    \end{theorem}
    \begin{corollary}
        Nếu bậc của nhóm hữu hạn $G$ là số nguyên tố $p$ thì bậc của nhóm con $H$ là $1$ hoặc $p$.

        Nếu bậc của $H$ là $p$, mọi điểm thuộc $G$ đều là phần tử sinh của nhóm con $H$
    \end{corollary}
\end{frame}
\subsection{Logarit rời rạc và mã hóa}
\begin{frame}{Bài toán logarit rời rạc đối với đường cong elliptic}
    \begin{definition}
        \label{define:3.1}
        Cho đường cong elliptic $E$ trên trường hữu hạn $F$, và nhóm $G$ bao gồm tập các điểm trên $E$ được định nghĩa với phép $+$. Cho điểm
        $P, Q \in G$ sao cho $Q$ là bội của $P$, hay $Q = kP$ với $k$ là số nguyên. Bài toán logarit rời rạc được định nghĩa như sau:

        \begin{tabular}{l}
            \\
            Cho trước 2 điểm $P, Q \in G$.                \\
            Tìm số nguyên $k$ nhỏ nhất thỏa mãn $kP = Q$. \\
            \\
        \end{tabular}

        $k$ gọi là logarit rời rạc của $Q$ theo cơ sở $P$.

    \end{definition}
\end{frame}

\begin{frame}{Bài toán logarit rời rạc đối với đường cong elliptic}
    Xét trường hữu hạn $\mathbb{F}_p$ bao gồm các số nguyên theo modulo $p$, trong đó $p$ là số nguyên tố.
    Kí hiệu $E(\mathbb{F}^2_p)$ là tập điểm của $\mathbb{F}^2_p$ nằm trên đường cong $E$.

    Điểm $P=(x,y) \in \mathbb{F}_p^2$ khi và chỉ khi, với mọi $a,b \in \mathbb{F}_p$, biểu thức được thỏa mãn:
    \begin{equation}
        \label{equation:3.8}
        y^2 \equiv x^3 + ax + b \pmod{p}
    \end{equation}

    $E(\mathbb{F}^2_p)$ là một nhóm abel với phép toán $+$ được định nghĩa ở trên.


\end{frame}
\begin{frame}{Bài toán logarit rời rạc đối với đường cong elliptic}
    Xét trường hữu hạn $\mathbb{F}_p$ bao gồm các số nguyên theo modulo $p$, trong đó $p$ là số nguyên tố.
    Kí hiệu $E(\mathbb{F}^2_p)$ là tập điểm của $\mathbb{F}^2_p$ nằm trên đường cong $E$.

    Điểm $P=(x,y) \in \mathbb{F}_p^2$ khi và chỉ khi, với mọi $a,b \in \mathbb{F}_p$, biểu thức được thỏa mãn:
    \begin{equation}
        y^2 \equiv x^3 + ax + b \pmod{p}
    \end{equation}

    $E(\mathbb{F}^2_p)$ là một nhóm abel với phép toán $+$ được định nghĩa ở trên.

    $$ \mid G \mid = \mid E(\mathbb{F}^2_p) \mid = h \cdot k,$$
    trong đó $k$ là số nguyên tố và là bậc của nhóm sinh bới $P$, $h$ nên nhỏ.
\end{frame}

\begin{frame}
    \fontsize{10pt}{12pt}
    \selectfont
    \frametitle{Thuật toán tìm tham số đường cong trên trường $\mathbb{F}_p$}

    \begin{algorithm}[H]
        \Input {$p$}
        \Output{$out = (p, a, b, P, k, h)$}
        \Begin
        {
            1. Chọn ngẫu nhiên đường cong $E$, hay ngẫu nhiên $a,b$  \\
            2. Dùng thuật toán Schoof để tìm bậc của $G$ \\
            3. Tìm ước nguyên tố $k$ lớn của $\mid G \mid$. Đảm bảo $h$ đủ nhỏ và $k$ đủ lớn với $h \cdot k = \mid G \mid$ \\
            4. \If{$k$ không đủ lớn}{
                Trở lại bước 1, tìm đường cong mới.
            }
            5. Chọn ngẫu nhiên $P \in G$ và tính $kP$ bằng thuật toán nhân đôi và cộng. \\
            6. \If{$kP \neq 1$}{
                $k$ không phải bậc của nhóm con sinh bởi $P$, trở lại bước 6 tìm điểm $P$ khác.
            }
            7. \Return $(p, a, b, P, k, h)$
        }
        \caption{Tìm tham số đường cong}
    \end{algorithm}

\end{frame}

\begin{frame}
    \frametitle{Một số đường cong đặc biệt}
    \begin{itemize}
        \item Đường cong Koblitz trên trường nhị phân:
              $$ y^2 + xy = x^3 + ax^2 + 1 \text{ với } a \in \{0,1\} \text{ trên trường $GF(2^m)$} $$ với $m$ là số nguyên tố
        \item Đường cong Weierstrass trên trường số nguyên tố:
              $$ y^2 = x^3 + ax + b$$ với $a,b \in \mathbb{F}_p$
              Được cho rằng đem lại hiệu quả bảo mật cao hơn đường Koblitz hay đường nhị phân dù chi phí tính toán tốn kém hơn.
    \end{itemize}

    NIST đã cung cấp một vài đường cong để sử dụng trong ECC. 3 trường $\mathbb{F}_p$ với các số $p$ cụ thể: $2^{607}-1, 2^{521}-1, 180*(2^{127}-1)^2+1$, với độ bảo mật tương ứng giảm dần.

    % Các đường cong này cho phép tính toán một cách hiệu quả và có thể sử dụng trong thực tế. Nhưng tính bảo mật của những đường cong này có thể bị đe dọa
\end{frame}
\begin{frame}
    \frametitle{Trao đổi khóa Diffe-Hellman}
    % \fontsize{10pt}{12pt}
    \begin{center}

        \begin{tabular}{|c|c|}
            \hline
            \multicolumn{2}{|c|}{Tạo tham số công khai}                              \\
            \hline
            \multicolumn{2}{|c|}{Cả hai chấp nhận cùng sử dụng $(p, a, b, P, k, h)$} \\
            % \multicolumn{2}{|c|}{và một số nguyên $g$ có bậc nguyên tố lớn trong $\mathbb{F}^*_p$}                                        \\
            \hline
            \hline
            \multicolumn{2}{|c|}{Thực hiện tính toán bí mật}                         \\
            \hline
            Alice                           & Bob                                    \\
            \hline
            Chọn bí mật một số nguyên $n_A$ & Chọn bí mật một số nguyên $n_B$        \\
            Tính $Q_A = n_AP$               & Tính $Q_B = n_BP$                      \\
            \hline
            \hline
            \multicolumn{2}{|c|}{Trao đổi giá trị công khai}                         \\
            \hline
            Alice gửi $Q_A$ cho Bob         & $\longrightarrow Q_A$                  \\
            $Q_B \longleftarrow$            & Bob gửi $Q_B$ cho Alice                \\
            \hline
            \hline
            \multicolumn{2}{|c|}{Tiếp tục thực hiện tính toán bí mật}                \\
            \hline
            Alice                           & Bob                                    \\
            \hline
            Tính $n_AQ_B \pmod{p}$          & Tính $n_BQ_A \pmod{p}$                 \\
            \hline
            \multicolumn{2}{|c|}{Khóa bí mật của cả hai là $n_An_BP$ }               \\
            \hline
        \end{tabular}

    \end{center}

\end{frame}

\begin{frame}
    \frametitle{Chữ ký số}
    \begin{center}
        \fontsize{10pt}{12pt}

        \begin{tabular}{|c|}
            \hline
            Alice muốn gửi tin nhắn $m$ cho Bob                      \\
            \hline
            Cả hai chấp nhận cùng sử dụng $(p, a, b, P, k, h)$       \\
            \hline
            $n_A$, $Q_A$ tương ứng là khóa bí mật và công khai Alice \\
            \hline
            \hline
            Alice chọn một số $n$ làm khóa tạm thời                  \\
            \hline
            Alice tính $Q = nP$,                                     \\
            đặt $r$ là tọa độ $x$ của điểm $Q$                       \\
            Tính $s = n^{-1}(m+ rn_A)$                               \\
            Gửi cặp ($r, s$) cho Bob                                 \\
            $\longrightarrow$                                        \\
            \hline
            Bob tính $u_1 = s^{-1}m \pmod{k}$                        \\
            $u_2 = s^{-1}r \pmod{k}$                                 \\
            $Q = u_1P + u_2Q_A$                                      \\
            kiểm tra nếu $r = x_Q$ thì đúng là Alice.                \\
            \hline
        \end{tabular}

    \end{center}


\end{frame}
\section{Classical and Quantum attacks on ECC}
\begin{frame}
    \frametitle{Bài toán của Sony}
    \begin{proposition}
        \label{pro:4.1}
        Nếu $n$ trong thuật toán chữ ký số không đổi, ta có thể khôi phục khóa bí mật từ chữ ký của Alice.
    \end{proposition}
    \begin{proof}
        Ta có 2 bộ chữ ký của Alice là $(r_1, s_1)$ và $(r_2, s_2)$.
        Vì $r = x_Q$ mà $Q = nP$, nên với $n$ không đổi, ta có $r_1 = r_2$.
        Lại có;
        $$ s_1 - s_2 = n^{-1}(m_1+rn_A) - n^{-1}(m_2+rn_A) = n^{-1}(m_1-m_2) \pmod{k}$$
        nên $$ n = (m_1 - m_2)(s_1 - s_2)^{-1} \pmod{k}$$
        Sau khi tính được $n$, ta tính được $$n_A = r^{-1}(s_1n-m_1) \pmod{k}$$
    \end{proof}
\end{frame}

\begin{frame}
    \frametitle{Tấn công Pollard Rho}

    \begin{algorithm}[H]
        \Input {$P, Q$}
        \Output{$out = (a,b,c,d)$}
        \Begin
        {
            Sinh ra dãy $S$, mỗi phần tử là một cặp $(a,b)$ \\
            Khởi tạo con trỏ $it_1 = 0$ và $it_2 = 0$ \\
            \While{$aP + bQ \neq cP + dQ$}
            {
                $it_1 \gets it_1 + 1$ \\
                $it_2 \gets it_2 + 1$ \\
                $(a,b) \gets S[it_1]$ \\
                $(c,d) \gets S[it_2]$ \\
            }

            \Return $(a,b,c,d)$
        }
        \caption{Tìm chu trình của Floyd}
    \end{algorithm}

\end{frame}
\begin{frame}
    \frametitle{Tấn công Pollard Rho}

    \begin{algorithm}[H]
        \Input {$P, Q$}
        \Output{$n$ thỏa mãn $nP = Q$}
        \Begin
        {
            $(a,b,c,d) \gets Floyd(P,Q)$ \\
            $n \gets (a-c)(d-b)^{-1}$ \\

            \Return $n$
        }
        \caption{Pollard Rho}
    \end{algorithm}
    Độ phức tạp thời gian: $O(\sqrt{k})$ hoặc $2^{O(b)}$ với $b$ là số bit của $k$.
\end{frame}
\section{Conclusion}


\begin{frame} [allowframebreaks]\frametitle{References}

    \bibliographystyle{apalike}
    \bibliography{bibfile}
\end{frame}

\end{document}